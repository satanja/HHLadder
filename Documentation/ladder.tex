\documentclass{article}
\usepackage[utf8]{inputenc}
\usepackage{a4wide}
\usepackage{multicol}
\title{Hoc Habet Ladder Competition}
\author{}
\date{}
\begin{document}
\maketitle
\section{Introduction}
Coming Tuesday Hoc Habet will organize a ladder competition, lasting until the end of the period\footnote{Bijvoorbeeld kwartiel (liever niet), semester, of jaar.}. Participation is not required, but it is completely without cost.\footnote{Lage entree barrière.} The idea of this competition is to create a more competitive environment in Hoc Habet throughout the year. The system is based on challenges, you can challenge the person above in the ranking to a \emph{ladder bout}. Each weapon has a ranking list, a so called \emph{ladder}, which should be updated after each ladder bout (see the rules). 

\section{Rules}
\subsection{General rules}
\begin{enumerate}
    \item If you are not on the ladder and you want to compete, you will be put at the bottom of the ladder.
    \item You may participate in multiple weapons. Add yourself to the ladders.
\end{enumerate}
\subsection{Challenge rules}
\begin{enumerate}
   \item You may only challenge fencers one position higher on the ladder of your preferred weapon.
   \item When challenging an opponent, it must be clear it is for the ladder competition, either from context or it is explicitly mentioned before the lader bout starts.
   \item You may decline at most 2 challenges (separate per weapon). If you decline the 3rd challenge, you will drop one position in the ladder, after which the count resets. This number tracked in a table, see Table \ref{tab:challenges}.
   \item If you are absent on a training, 1 declined challenge will be added to your current total, and if applicable, your position will be adjusted.
\end{enumerate}
\subsection{Ladder bout rules}
\begin{enumerate} 
    \item Ladder bouts may be fenced mechanically, but fencing electrically is a lot more convenient. As long as both parties agree on the result, it is fine.
    \item Fencers may follow KNAS tournament bout rules, but it is not necessary, i.e. referees, cards, priority, passivity, etc. may be ignored.
    \item Fencers may decide the number of touches required to win the ladder bout between themselves, with a minimum of 5 touches.
    \item If the lower ranked fencer beats the higher ranked fencer, the fencers will change positions on the ladder.
    \item If the higher ranked fencer beats the lower ranked fencer, the ladder rankings of both fencers are not affected.
    \item If the ladder has to be changed, this should be done directly after the ladder bout.
\end{enumerate}
\section{Examples}
\subsection{Ladders}
Consider the ladder for foil (Table \ref{tab:foil}).
\begin{table}[ht]
    \centering
    \begin{tabular}{|c|c|}
        \hline
        Ranking & Fencer \\ \hline
        1 & Alice \\ \hline
        2 & Bob \\ \hline
        3 & Charlie \\ \hline
    \end{tabular}
    \caption{Foil ladder before}
    \label{tab:foil}
\end{table}
Charlie challenges Bob to a ladder bout of 10 touches, and he accepts. Charlie wins the ladder bout, and so the ladder will be modified to Table \ref{tab:foil2}:
\begin{table}[ht]
    \centering
    \begin{tabular}{|c|c|}
        \hline
        Ranking & Fencer \\ \hline
        1 & Alice \\ \hline
        2 & Charlie \\ \hline
        3 & Bob \\ \hline
    \end{tabular}
    \caption{Foil ladder after}
    \label{tab:foil2}
\end{table}
\subsection{Declined challenges}
\begin{table}[ht]
    \centering
    \begin{tabular}{|c|c|}
        \hline
        Fencer & Declined \\ \hline
        Alice & 1 \\ \hline
        Bob & 2 \\ \hline
        Charlie & 0 \\ \hline
    \end{tabular}
    \caption{Declined challenges foil}
    \label{tab:challenges}
\end{table}
\section{Prizes}
Willen we dit doen?
\end{document}