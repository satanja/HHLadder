\documentclass{article}
\usepackage[utf8]{inputenc}
\usepackage{a4wide}
\usepackage{multicol}
\title{Hoc Habet Ladder Competition \\ 2018 - 2019}
\author{}
\date{}
\begin{document}
\maketitle
\section{Introduction}
This Tuesday Hoc Habet will organize a ladder competition every semester, a so called \emph{season} (See section \ref{sec:Seasons} for explanation about the seasons). Participation is not required, but it is completely without cost. The idea of this competition is to create a more competitive environment in Hoc Habet throughout the year. The system is based on challenges, you can challenge the person above in the ranking to a \emph{ladder bout}. Each weapon has a ranking list, a so called \emph{ladder}, which should be updated after each ladder bout (see the rules). There will also be a specific foil ladder for beginning fencers, this ladder is only accessible by fencers that have fenced (in total) a year or less. After each season a winner will be crowned for each ladder.

\section{Rules}
\paragraph{Important note} In case of doubt, a member of the board should be informed and will make an appropriate, irrefutable decision. If the dispute involves a member of the board, a different member of the board should be informed.
\subsection{General rules}
\begin{enumerate}
    \item The ladder ends 2 weeks before the end of the semester (before the exam weeks start).
    \item On signup, everyone is added on their preferred ladder in alphabetical order (first name). 
    \item If you are not on the ladder and you want to compete after the signup, you will be put at the bottom of the ladder, regardless of the correct alphabetical order.
    \item You may participate in multiple weapons. Add yourself to the ladders.
    \item If you are absent more than 4 weeks, the board may decide to put you at the bottom of the ladder.
    \item If a board member is him/herself part of a dispute about any part of the ladder. He/She may not make any decisions about its outcome and has to let another board member decide.
\end{enumerate}
\subsection{Challenge rules}
\begin{enumerate}
    \item You may only challenge fencers higher than yourself on the ladder of your preferred weapon.
    \item When challenging an opponent, it must be clear it is for the ladder competition, either from context or it is explicitly mentioned before the ladder bout starts.
\end{enumerate}
\subsection{Ladder bout rules}
\begin{enumerate}
    \item Ladder bouts may start when the free fencing starts (21:30), you may challenge your opponent before though.
    \item Ladder bouts may be fenced mechanically, but fencing electrically is a lot more convenient. As long as both parties agree on the method, it is fine.
    \item Fencers may follow KNAS tournament bout rules, but it is not necessary, i.e. referees, cards, passivity, etc. may be ignored.
    \item Fencers may decide the number of touches required to win the ladder bout between themselves, with a minimum of 5 touches.
    \item If the lower ranked fencer beats the higher ranked fencer, the fencers will change positions on the ladder.
    \item If the higher ranked fencer beats the lower ranked fencer, the ladder rankings of both fencers are not affected.
    \item If the ladder has to be changed, this should be done directly after the ladder bout.
\end{enumerate}
\section{Seasons} \label{sec:Seasons}
The year 2018-2019 is divided in two seasons, see Table \ref{tab:dates}.
\begin{table}[ht]
    \centering
    \begin{tabular}{|c|c|c|}
        \hline
        Season \# & Start date & End date \\ \hline
        Season 1 & 23-10-2018 & 15-1-2019 \\ \hline
        Season 2 & 5-2-2019 & 18-6-2019 \\ \hline
    \end{tabular}
    \caption{Start and end dates}
    \label{tab:dates}
\end{table}
At the end of each season for each ladder a winner will be crowned for each ladder, and receive a small prize\footnote{This prize is chosen each season at discretion of the board.}. Since you are able to sign up for multiple ladders, you can win multiple times as well, e.g. for foil and epee.
\section{Ladder examples}
Consider the ladder for foil (Table \ref{tab:foil}).
\begin{table}[ht]
    \centering
    \begin{tabular}{|c|c|}
        \hline
        Ranking & Fencer \\ \hline
        1 & Alice \\ \hline
        2 & Bob \\ \hline
        3 & Charlie \\ \hline
    \end{tabular}
    \caption{Foil ladder before}
    \label{tab:foil}
\end{table}
Charlie challenges Alice to a ladder bout of 10 touches, and she accepts. Charlie wins the ladder bout, and so the ladder will be modified to Table \ref{tab:foil2}:
\begin{table}[ht]
    \centering
    \begin{tabular}{|c|c|}
        \hline
        Ranking & Fencer \\ \hline
        1 & Charlie \\ \hline
        2 & Bob \\ \hline
        3 & Alice \\ \hline
    \end{tabular}
    \caption{Foil ladder after}
    \label{tab:foil2}
\end{table}
\end{document}